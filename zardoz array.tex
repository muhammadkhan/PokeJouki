\documentclass[10pt]{article}
\usepackage[margin=.75in]{geometry}
\usepackage{mathtools}
\DeclarePairedDelimiter{\ceil}{\lceil}{\rceil}
\begin{document}
\noindent
Arjun Biddanda (aab227)\\
Muhammad Khan (mhk98)\\
CS 3110 - Fall 2012\\
2:30 PHL Discussion\\
11/30/12\\
\begin{center}
\section*{Problem Set 6}
\subsection*{Part 9: Analysis of Algorithms for the Zardoz Array}
\end{center}
\subsubsection*{1) SEARCH}
For a collection of $n$ elements, the zardoz array consists of $k$ sorted arrays, where $k \equiv
\ceil{\log_2(n+1)}$, such that the binary representation of $n$ is $n_{k-1}n_{k-2}\cdots n_1n_0$ and each sorted array $A_i$ is empty if $n_i = 0$ and has $2^i$ elements if $n_i = 1$. Performing a complexity analysis on the search operation for this data structure, we see that in the worst case, the element we seek is in the $k$th sorted array, which will require $O(k)$ time to reach. In this array, performing a binary search would take $O(log_2 2^i) = O(i)$ time. In the $k$th sorted array, $i = k - 1$, thus the binary search algorithm also takes $O(k)$ time. Thus, together, in the worst case, searching in a zardoz array takes $O(k^2)$ time.
\subsubsection*{2) INSERT}
When inserting an element there are three cases which must be looked at:\\
\begin{center}
1) $\cdots$ 0 $\rightarrow$ $\cdots$ 1\\
2) $\cdots$ 01 $\rightarrow \cdots$ 10\\
3) 111$\cdots$ 111 $\rightarrow$ 100$\cdots$ 001\\
\end{center}
Case (1) corresponds to the situation when the binary representation of $n$ ends in a 0, which means that the binary representation of $n+1$ is equivalent to that of $n$ except this last 0 is replaced by a 1. The original zardoz array (before insertion) will have $A_0$ empty and $A_1$ filled with $2^1 = 2$ elements. To insert a new element, we simply set the only element of $A_0$ to be this newly inserted element, and we have a new zardoz array. This operation takes $O(1)$ time.

Case (2) corresponds to the situation where the last two digits of the binary representation of $n$ are 01. This translates to a zardoz array where $A_0$ is full with one element and $A_1$ is empty. If $A_1$ were to be full, it would have to have two elements. When inserting an element $a$ into a zardoz array of this nature, we simply take $a$ and the element of $A_0$ and put them into one array which becomes the new $A_1$. Sorting this two-element array is a constant-time ($O(1)$) operation.

Case (3), likewise, corresponds to the situation when the entire binary representation is a string of 1s. For any $n$ in this case, there are $k$ 1s. Adding one element to this collection will result in $n+1$ elements, which has a binary representation of 100$\cdots$001 ($k-1$ zeroes), which means the only filled sorted arrays are $A_0$ and $A_k$. These arrays will be formed by taking the newly added element to fill up the new $A_0$ (constant time operation). The remaining $n$ elements are sorted amongst each other to fill up $A_k$. This stage will be implemented using the heapsort algorithm, which in its worst case is linearithmic: $O(n\log n)$. Thus, in this case only, insertion takes $O(n\log n)$ time.

To analyze amortized complexity, we will use the potential method, with a potential function given by $$\Phi(n) = n - k$$ with $n$ the number of elements and $k$ as defined previously. It is easy to see that $\Phi(0) = 0$. During the fast operations (cases 1 and 2), the value of $k$ remains unchanged, so the amortized time for a fast operation is $$c + \Phi(n + 1) - \Phi(n) = 1 + (n + 1) - k - (n - k) = 2$$ The substitution $c = 1$ arises from the fact that a fast operation is $O(1)$.

Likewise, during a slow operation, $k$ increases by 1 after the insertion, so the amortized time is $$c + \Phi(n + 1) - \Phi(n) = n\log n + (n + 1) - (k + 1) - (n - k) = n\log n$$
After a series of insertions, the value of $n$ gradually increases, so the frequency of slow operations decreases. Thus, the cost of insertion is dominated by the cost of fast operations. Thus, insertion takes amortized constant time $O(1)$ which satisfies the $o(n)$ condition.
\end{document}